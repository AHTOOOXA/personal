\documentclass[letterpaper,11pt]{article}

\usepackage{latexsym}
\usepackage[empty]{fullpage}
\usepackage{titlesec}
\usepackage{marvosym}
\usepackage[usenames,dvipsnames]{color}
\usepackage{verbatim}
\usepackage{enumitem}
\usepackage[hidelinks]{hyperref}
\usepackage{fancyhdr}
\usepackage[russian]{babel}
\usepackage{tabularx}
\input{glyphtounicode}

\pagestyle{fancy}
\fancyhf{} % clear all header and footer fields
\fancyfoot{}
\renewcommand{\headrulewidth}{0pt}
\renewcommand{\footrulewidth}{0pt}

% Adjust margins
\addtolength{\oddsidemargin}{-0.5in}
\addtolength{\evensidemargin}{-0.5in}
\addtolength{\textwidth}{1in}
\addtolength{\topmargin}{-.5in}
\addtolength{\textheight}{1.0in}

\urlstyle{same}

\raggedbottom
\raggedright
\setlength{\tabcolsep}{0in}

% Sections formatting
\titleformat{\section}{
  \vspace{-12pt}\scshape\raggedright\large
}{}{0em}{}[\color{black}\titlerule \vspace{-5pt}]

\pdfgentounicode=1

\newcommand{\resumeItem}[1]{
  \item\small{
    {#1 \vspace{-3pt}}
  }
}

\newcommand{\resumeSubheading}[4]{
  \vspace{-1pt}\item
    \begin{tabular*}{0.97\textwidth}[t]{l@{\extracolsep{\fill}}r}
      \textbf{#1} & #2 \\
      \textit{\small#3} & \textit{\small #4} \\
    \end{tabular*}\vspace{-8pt}
}

\newcommand{\resumeProjectHeading}[2]{
    \item
    \begin{tabular*}{0.97\textwidth}{l@{\extracolsep{\fill}}r}
      \small#1 & #2 \\
    \end{tabular*}\vspace{-8pt}
}

\newcommand{\resumeSubHeadingListStart}{\begin{itemize}[leftmargin=0.15in, label={}]}
\newcommand{\resumeSubHeadingListEnd}{\end{itemize}}
\newcommand{\resumeItemListStart}{\begin{itemize}}
\newcommand{\resumeItemListEnd}{\end{itemize}\vspace{-6pt}}

\begin{document}
\vspace*{-50pt}

\begin{center}
    \textbf{\huge \scshape Антон Сафонов} \\ \vspace{-1pt}
    \large Python Backend Developer \\ \vspace{-1pt}
    \small \href{mailto:anton.safonov.loves.cats@gmail.com}{\underline{anton.safonov.loves.cats@gmail.com}} $|$ 
    \href{https://t.me/anton_whatever}{Telegram: \underline{@anton\_whatever}} 
    % $|$
    % \href{https://github.com/AHTOOOXA}{GitHub}
\end{center}

\section{Обо мне}
\small\begin{itemize}[leftmargin=0.15in, label={}]
\item{Backend-разработчик с опытом проектирования масштабируемых API, ETL-конвейеров и микросервисных систем. Уверенно работаю с продакшен-нагрузкой, DevOps-инструментами и аналитическими платформами. Люблю разбираться в сложных системах, писать чистый код и доводить задачи до стабильного результата. Погружаюсь в бизнес-домен, понимаю требования фронтенда и продукта, что помогает строить целостные решения. Готов как к глубокой технической работе, так и к взаимодействию с командами и заказчиками.}
\end{itemize}
%-----------SKILLS-----------
\section{Технические Навыки}
 \begin{itemize}[leftmargin=0.15in, label={}]
    % \small{\item{
    %  \textbf{Языки}{: Python, JavaScript, TypeScript, SQL} \\
    %  \textbf{Технологии}{: FastAPI, Django, Vue.js, HTMX} \\
    %  \textbf{Базы данных}{: PostgreSQL, Clickhouse, Redis, Elasticsearch} \\
    %  \textbf{Инструменты и DevOps}{: Docker, Git, CI/CD, RabbitMQ, Prometheus, Grafana, Loki, Posthog, Yandex Cloud} \\
    %  \textbf{Методологии}{: Микросервисы, ETL, API разработка, Agile, Scrum}
    % }}
    \item Python, FastAPI, Django, SQLAlchemy, PostgreSQL, ClickHouse, Redis,  RabbitMQ, Elasticsearch (ELK), Docker, Git, CI/CD, Prometheus, Grafana, Loki, Sentry, Yandex Cloud, arq, асинхронность, asyncio, микросервисы, REST API, JavaScript, TypeScript, Vue.js, HTMX, Tailwind CSS, PostHog
 \end{itemize}


%-----------EXPERIENCE-----------
\section{Опыт Работы}
  \resumeSubHeadingListStart
    \resumeSubheading
      {Python Backend Developer}{Март 2024 - Апрель 2025}
      {Realytics (AI-аналитика производительности для потребительских бизнесов)}{Никосия, Кипр (Удалённо)}
      \resumeItemListStart
        \resumeItem{Работал над rule-based движком агрегации организаций по брендам}
        \resumeItem{Спроектировал и реализовал API микросервис обработки и очистки адресов при помощи LLM и микросервис локализации}
        \resumeItem{Создал высокопроизводительный ETL-конвейер для агрегации и обработки данных из множества сложных источников}
        \resumeItem{Реализовал продуманный API для B2B-клиентов с приоритетом на расширяемость, кастомизируемость, возможность аудита и отслеживания историчности данных}
        % \resumeItem{Python, FastAPI, Clickhouse, PostgreSQL, Redis, RabbitMQ, Docker, Prometheus, Grafana}
      \resumeItemListEnd
      \small\vspace{-3pt}\begin{itemize}[leftmargin=0.15in, label={}]
    \item{\textbf{Стек:} Python, FastAPI, Clickhouse, PostgreSQL, Redis, RabbitMQ, Docker, Prometheus, Grafana}
    \end{itemize}
    \vspace{-8pt}
    \resumeSubheading
      {Python Backend Developer}{Январь 2022 - Март 2024}
      {e-comet.io (сервис аналитики и автоматизации продвижения на Wildberries)}{Москва, Россия}
      \resumeItemListStart
        \resumeItem{Оптимизировал высоконагруженную аналитическую часть API, снизив нагрузку на Clickhouse на 20\% и время ответа API на 40\% за счёт оптимизации запросов и архитектурных изменений}
        \resumeItem{Разработывал rule-based движок автоматического управления рекламой с гибкой настройкой и интеграцией с маркетплейсами}
        \resumeItem{Разрабатывал ключевые модули приложения: аналитический функционал, AI ответы на отзывы, платежи, сбор данных}
        \resumeItem{Реализовал fail-safe-логику для всех критичных бизнес-операций, связанных с управлением ценами и бюджетами клиентов, включая обработку недоступности API маркетплейсов и внутренних сервисов}
        \resumeItem{Работал в тесной связке с фаундерами для реализации кастомных высокоэффективных решений для корпоративных клиентов}
      \resumeItemListEnd
      \small\vspace{-3pt}\begin{itemize}[leftmargin=0.15in, label={}]
        \item{\textbf{Стек:} Python, FastAPI, Django, Clickhouse, PostgreSQL, Redis, RabbitMQ, Docker, Elasticsearch, Yandex Cloud}
        \end{itemize}
  \resumeSubHeadingListEnd

%-----------PROJECTS-----------
\section{Проекты}
    \resumeSubHeadingListStart
    % \resumeProjectHeading
    %     {\textbf{AI Tarot - Telegram Mini App - \href{https://t.me/tarotmeowbot/app?startapp=r-resume}{@TarotMeowBot}}}{2024}
    %     \resumeItemListStart
    %         \resumeItem{Разработал Telegram Mini App для AI-предсказаний Таро, спроектировав и реализовав fullstack веб-приложение на стеке FastAPI + Vue.js.}
    %         \resumeItem{Реализовал ключевые модули: аутентификация, локализация, реферальная программа, система уведомлений, аналитика и интеграция оплаты.}
    %         \resumeItem{Спроектировал монорепозиторий с ботом, воркером и API, выстроив трёхслойную модульную архитектуру.}
    %         \resumeItem{Использовал Pydantic совместно с OpenAPI (openapi-typescript и openapi-fetch) для типизации, а также arq для обработки фоновых задач.}
    %         \resumeItem{ FastAPI, aiogram, arq, SQLAlchemy, PostgreSQL, Redis, Typescript, Vue, Pinia, openapi-typescript, Tailwind, Docker, Dokploy, PostHog}
    %     \resumeItemListEnd
    
    
    \resumeProjectHeading
        {\textbf{AI Tarot - Telegram Mini App \underline{\href{https://t.me/tarotmeowbot/app?startapp=r-resume}{@TarotMeowBot}}}}{2024}
        \small\vspace{-8pt}\begin{itemize}[leftmargin=0.15in, label={}]
        \item{Разработал AI-продукт с нуля как единственный инженер и продуктовый владелец: спроектировал архитектуру, реализовал backend, frontend и тг-бота, внедрил платёжную систему, аналитику, уведомления, локализацию, реферальную программу, CI/CD. Привлёк 700 пользователей, добился retention первого дня/недели/месяца 59\%/33.8\%/23\%. Проходили онбординг 92\%, тратили бесплатный лимит 52\%, более 5\% платили. Принимал все продуктовые и UX-решения на основе аналитикии и интервью.}
        \end{itemize}
        \small\vspace{-8pt}\begin{itemize}[leftmargin=0.15in, label={}]
        \item{\textbf{Стек:} FastAPI, aiogram, arq, SQLAlchemy, Redis, Typescript, Vue, Pinia, Tailwind, Docker, PostHog}
        \end{itemize}

      % \resumeProjectHeading
      %     {\textbf{Система планирования промышленного производства}}{2023}
      %     \resumeItemListStart
      %       \resumeItem{Разработал корпоративное веб-приложение для управления производственными процессами для 170 сотрудников}
      %       \resumeItem{Реализовал функции: динамические вложенные формы, управление доступом на основе ролей, аналитика в реальном времени, drag and drop интерфейс планирования}
      %       \resumeItem{Сократил время планирования загрузки производственных мощностей более чем на 200 часов в месяц}
      %       \resumeItem{Python, Django, HTMX, Alpine.js, Bootstrap, Grafana, Loki, Sentry, Dokploy}
      %     \resumeItemListEnd

    \vspace{-8pt}
    \resumeProjectHeading
        {\textbf{Система планирования промышленного производства}}{2023}
        \small\vspace{-8pt}\begin{itemize}[leftmargin=0.15in, label={}]
        \item{Спроектировал и самостоятельно разработал веб-приложение для управления производственными процессами на предприятии с 200+ сотрудниками. Реализовал динамические вложенные формы, drag-and-drop интерфейс планирования, RBAC и real-time аналитику. Решение позволило сократить планирование на 200+ часов в месяц. Выяснял требования и работал напрямую со стейкхолдерами }
        \end{itemize}
        \small\vspace{-8pt}\begin{itemize}[leftmargin=0.15in, label={}]
        \item{\textbf{Стек:} Django, HTMX, Alpine.js, Bootstrap, Grafana, Loki, Sentry, Docker}
        \end{itemize}
    \resumeSubHeadingListEnd

%-----------EDUCATION-----------
\section{Образование}
  \resumeSubHeadingListStart
    \resumeSubheading
      {НИУ ВШЭ}{2020 -- 2024}
      {Бакалавриат - Прикладная Математика и Информатика}{Факультет компьютерных наук}
  \resumeSubHeadingListEnd
\end{document}
